%==============================================================================
% Sjabloon onderzoeksvoorstel bachelorproef
%==============================================================================
% Gebaseerd op LaTeX-sjabloon ‘Stylish Article’ (zie voorstel.cls)
% Auteur: Jens Buysse, Bert Van Vreckem

\documentclass[fleqn,10pt]{voorstel}

%------------------------------------------------------------------------------
% Metadata over het voorstel
%------------------------------------------------------------------------------

\JournalInfo{HoGent Bedrijf en Organisatie}
\Archive{Bachelorproef 2018 - 2019} % Of: Onderzoekstechnieken

%---------- Titel & auteur ----------------------------------------------------

% TODO: geef werktitel van je eigen voorstel op
\PaperTitle{De zin en onzin van SEO tools: welke kunnen als hulpmiddel gebruikt worden voor meer zichtbaarheid in zoekmachines voor KMO's en hoe gebeurt het verbeteren van een keyword tool?}
\PaperType{Onderzoeksvoorstel Bachelorproef} % Type document

% TODO: vul je eigen naam in als auteur, geef ook je emailadres mee!
\Authors{Nathan Verduyn\textsuperscript{1}} % Authors
\CoPromotor{Frederik Vermeire\textsuperscript{2} (WISEO)}
\affiliation{\textbf{Contact:}
  \textsuperscript{1} \href{mailto:nathan.verduyn.w1611@student.hogent.be}{nathan.verduyn.w1611@student.hogent.be};
  \textsuperscript{2} \href{mailto:frederik@wiseo.be}{frederik@wiseo.be};
}

%---------- Abstract ----------------------------------------------------------

\Abstract{Op dit moment is er nog geen compleet en waterdicht onderzoek gedaan naar een vergelijking tussen SEO tools en het resultaat dat ze opleveren. Dit moet worden onderzocht omdat WISEO hierdoor commercieel meer opdrachten kan binnenhalen door beter onderzoek te kunnen voeren aan een goedkoper tarief(adv. goedkopere tools). Er zal een onderzoek gedaan worden tussen verschillende SEO tools en er wordt een eigen SEO tool geschreven. In het document staat een vergelijkende studie tussen bepaalde tools met de volgende factoren: prijs, soort tool en de prestatie ervan. De verwachting van dit onderzoek is dat er goedkopere alternatieven zijn voor bepaalde SEO tools met dezelfde of betere werking die nu door WISEO worden gebruikt bij KMO's (middelgrote in alle sectoren). Door dit werk zal er meer duidelijkheid zijn welke SEO tools beter in de toekomst kunnen worden gebruikt. 
}

%---------- Onderzoeksdomein en sleutelwoorden --------------------------------
% TODO: Sleutelwoorden:
%
% Het eerste sleutelwoord beschrijft het onderzoeksdomein. Je kan kiezen uit
% deze lijst:
%
% - Mobiele applicatieontwikkeling
% - Webapplicatieontwikkeling
% - Applicatieontwikkeling (andere)
% - Systeembeheer
% - Netwerkbeheer
% - Mainframe
% - E-business
% - Databanken en big data
% - Machineleertechnieken en kunstmatige intelligentie
% - Andere (specifieer)
%
% De andere sleutelwoorden zijn vrij te kiezen

\Keywords{Onderzoeksdomein. E-business --- Google SEO --- SEO tools} % Keywords
\newcommand{\keywordname}{Sleutelwoorden} % Defines the keywords heading name

%---------- Titel, inhoud -----------------------------------------------------

\begin{document}

\flushbottom % Makes all text pages the same height
\maketitle % Print the title and abstract box
\tableofcontents % Print the contents section
\thispagestyle{empty} % Removes page numbering from the first page

%------------------------------------------------------------------------------
% Hoofdtekst
%------------------------------------------------------------------------------

% De hoofdtekst van het voorstel zit in een apart bestand, zodat het makkelijk
% kan opgenomen worden in de bijlagen van de bachelorproef zelf.
%---------- Inleiding ---------------------------------------------------------

\section{Introductie} % The \section*{} command stops section numbering
\label{sec:introductie}

Het probleem is bij vele SEO tools dat ze redelijk duur geprijsd zijn en er betere alternatieven bestaan op de markt. Hierin zal een vergelijking gemaakt worden tussen verschillende tools op vlak van prijs, soort tool en prestatie. De reden van dit onderzoek is omdat WISEO op die manier commercieel meer opdrachten kan binnenhalen door beter onderzoek te kunnen voeren aan een goedkoper tarief. De doelstelling is dus om de volgende onderzoeksvragen te beantwoorden: Welke seo tools kunnen best als hulpmiddel gebruikt worden voor meer zichtbaarheid in zoekmachines? (Hierbij wordt gekeken naar welke tools de meeste informatie weergeven zodat SEO'ers zoals WISEO dit kunnen implementeren) Wat is duur voor een SEO tool (vergelijkende studie)? Hoe gebeurt het uitwerken en verbeteren van een keyword tool?

%---------- Stand van zaken ---------------------------------------------------

\section{Literatuurstudie}
\label{sec:Literatuurstudie}

In dit domein zijn er al gelijkaardige onderzoeken te vinden in de vorm van blogposts. Om te beginnen wordt er gerefereerd naar een artikel over keywordplanners, \textcite{BKRT2018}. Met behulp van deze tools zullen de beste functionaliteiten (dit zal pas duidelijk zijn na een onderzoek) gekozen worden om dan een eigen zoekwoorden tool te maken.


Er worden in de referenties nog 2 andere artikels vermeld die gaan over de beste SEO tools, \textcite{SEO13}, en de volledige lijst van SEO tools 2018, \textcite{SEOCOMPLETE}. Deze blogposts hebben relevantie tot het onderzoek die hier zal gevoerd worden in de zin dat er ook een vergelijking zal gebeuren tussen meerdere tools. De meeste tools zijn trials die na verloop van tijd betalend zijn en duur kunnen oplopen. WISEO, als ervaren SEO specialisten, zegt hierbij dat er nog te weinig onderzoek wordt gedaan naar tools die gelijkaardig werken en een goedkoper (de studie zal ook uitwijzen wat goedkoop is en wat niet) alternatief vormen. 

% Voor literatuurverwijzingen zijn er twee belangrijke commando's:
% \autocite{KEY} => (Auteur, jaartal) Gebruik dit als de naam van de auteur
%   geen onderdeel is van de zin.
% \textcite{KEY} => Auteur (jaartal)  Gebruik dit als de auteursnaam wel een
%   functie heeft in de zin (bv. ``Uit onderzoek door Doll & Hill (1954) bleek
%   ...'')

%---------- Methodologie ------------------------------------------------------
\section{Methodologie}
\label{sec:methodologie}

Er zal een vergelijkende studie plaatsvinden tussen meerdere SEO tools die gemeten en ingedeeld zullen worden op basis van prijs, soort tool (er zijn tools met verschillende doeleinden zoals: keywordresearch, linkbuilding, ranking, content optimization en technical SEO tools) en de prestatie (wat levert de meeste en duidelijkste informatie op die SEO'ers het best kunnen gebruiken) ervan. Hieruit zal er vooral duidelijkheid blijken uit welke tools de meeste info kan worden gevonden per SEO onderdeel (zoekwoordenonderzoek, SEO audit opstellen, enz...). De experimenten zullen onder andere bestaan uit: zoekwoordenonderzoektools vergelijken om te kijken welke het beste resultaat levert, vergelijking bij audit tools om zoveel mogelijk info te weten te komen, enz... Voor het onderzoek en op vraag van WISEO wordt ook een eigen tool gemaakt op vlak van zoekwoordenonderzoek om aan te tonen dat er nog verbetering mogelijk is bij sommige tools. Er zullen voornamelijk tools getest en vergeleken worden die vermeld staan in de opgesomde artikelen bij referenties.

%---------- Verwachte resultaten ----------------------------------------------
\section{Verwachte resultaten}
\label{sec:verwachte_resultaten}

Als resultaat wordt er verwacht dat er tools een beter alternatief vormen voor bepaalde tools die momenteel meest gebruikt worden door KMO's en WISEO. Er moet een duidelijk beeld worden gevormd welke de beste zijn zodat WISEO hier gebruik van kan maken. Tussen de verschillende soorten tools (wat je ermee kan bereiken) wordt een vergelijking gemaakt waarbij rekening gehouden wordt met de kost en prestaties (welke het meeste en duidelijke info levert). Als laatste wordt er een eigen keyword research tool geschreven die in de toekomst gebruikt kan worden.

%---------- Verwachte conclusies ----------------------------------------------
\section{Verwachte conclusies}
\label{sec:verwachte_conclusies}

De verwachte conclusie is dat er een duidelijke lijst zal worden weergegeven tussen de prijs/kwaliteit van verscheidene tools. Verder wordt er verwacht dat er vele tools zijn met te weinig functionaliteit en deze dus beter vervangen worden door andere, betere SEO tools. In de artikelen (zie referenties) worden soms ook dure tools vermeld. Dit onderzoek zal duidelijk maken dat dit niet altijd hoeft. Als laatste conclusie wordt er verwacht dat het mogelijk is om een eigen keyword tool te implementeren die aantoont dat er nog mogelijkheid is tot verbetering op vlak van SEO tools.




%------------------------------------------------------------------------------
% Referentielijst
%------------------------------------------------------------------------------
% TODO: de gerefereerde werken moeten in BibTeX-bestand ``voorstel.bib''
% voorkomen. Gebruik JabRef om je bibliografie bij te houden en vergeet niet
% om compatibiliteit met Biber/BibLaTeX aan te zetten (File > Switch to
% BibLaTeX mode)

\phantomsection
\printbibliography[heading=bibintoc]

\end{document}
