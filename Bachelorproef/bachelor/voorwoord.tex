%%=============================================================================
%% Voorwoord
%%=============================================================================

\chapter*{Woord vooraf}
\label{ch:voorwoord}

%% TODO:
%% Het voorwoord is het enige deel van de bachelorproef waar je vanuit je
%% eigen standpunt (``ik-vorm'') mag schrijven. Je kan hier bv. motiveren
%% waarom jij het onderwerp wil bespreken.
%% Vergeet ook niet te bedanken wie je geholpen/gesteund/... heeft
Een bachelorproef maken is de laatste stap, naast stage, om een diploma te behalen en een eerste stap in de onderzoekswereld. Voor mij was het een unieke kans om na al die jaren studeren de theoretische kennis om te kunnen zetten in een eigen onderzoek.

SEO interesseerde mij al voor ik aan deze bachelorproef begon te schrijven. Het idee om verschillende tools met elkaar te vergelijken en zelf een tool te maken kwam vanuit mijn stageplek, WiSEO. Bij deze bachelorproef heb ik veel bijgeleerd over verschillende SEO-tools en hun werking. Zelf dacht ik niet dat je met tools zoveel resultaten kon zien. 

Uiteraard gaat zo'n onderzoek gepaard met vallen en opstaan, en heb je de steun/ervaring nodig van anderen, waarvoor mijn oprechte dank. 

Graag wil ik mijn promotor, Liesbeth Lewyllie, bedanken voor het nalezen van mijn onderzoek en dat nodige duwtje in de rug te geven. Daarnaast bedank ik ook graag mijn co-promotor, Frederik Vermeire, om mij veel bij te leren over SEO en ook alle nodige tools ter beschikking te stellen. 

Tenslotte wil ik nog mijn ouders bedanken. Zij hebben mij 3 jaar gesteund, zowel moreel als financieel, zowel in leuke als in moeilijke tijden. Daarom wil ik hen oprecht bedanken voor de kansen die ik gekregen heb. 