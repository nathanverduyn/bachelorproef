\IfLanguageName{english}{%
\selectlanguage{dutch}
\chapter*{Samenvatting}
\lipsum[1-4]
\selectlanguage{english}
}{}

%%---------- Samenvatting -----------------------------------------------------
% De samenvatting in de hoofdtaal van het document

\chapter*{\IfLanguageName{dutch}{Samenvatting}{Abstract}}

Search Engine Optimization (SEO) of zoekmachineoptimalisatie is een online marketing strategie waarmee je website makkelijker terug te vinden is in Google. Digitale marketing bedrijven gebruiken dit om meer bezoekers te genereren voor hun klanten en maken daarbij altijd gebruik van SEO-tools. Zo’n tools geven meer inzicht over de huidige posities in Google, wat er technisch (op SEO-vlak) niet goed zit bij een website, enz… .

Op dit moment is er nog geen compleet en waterdicht onderzoek gedaan naar een vergelijking tussen SEO-tools en het resultaat dat ze opleveren. Dit moet worden onderzocht omdat WISEO hierdoor commercieel meer opdrachten kan binnenhalen, door een beter SEO-onderzoek te kunnen uitvoeren, aan een goedkoper tarief (adv. goedkopere tools en tools die betere resultaten hebben). 

Het onderzoek bestaat uit twee delen: een vergelijkende studie tussen SEO-tools en het verbeteren van een bestaande zoekwoord tool.

In het eerste deel gebeurt er een literatuurstudie waarbij tools met enkele aspecten worden ingedeeld in categorieën en getest worden. De tools worden in 4 categorieën ingedeeld: zoekwoord tools, ranking tools, linkbuilding tools en technische tools. Per categorie zal er duidelijk zijn welke tool de beste is op vlak van prijs en prestatie. De experimenten gebeuren door middel van een vergelijking van zoekwoord onderzoek tools, ranking tools, linkbuilding tools en een vergelijking bij technische tools om zoveel mogelijk info te weten te komen over een website (kijken welke het beste resultaat levert). Als resultaat is het duidelijk wat de capaciteiten zijn van elke geteste tool waarbij voor elke categorie 1 of 2 tools het best gebruikt kunnen worden. De conclusie van deze literatuurstudie is dat er vaak al veel bereikt kan worden binnen SEO zonder het gebruik van betaalde tools. Het is ook zo dat de duurdere tools niet per se het beste en duidelijkste resultaat opleveren maar dat minder populaire (en goedkopere) tools een beter alternatief kunnen zijn. Door dit onderzoek is het duidelijk voor WiSEO welke tools ze kunnen blijven gebruiken of eventueel vervangen door andere tools. 

In het tweede deel van het onderzoek wordt uitgebreid besproken waarom er een keyword tool geprogrammeerd wordt en hoe deze tool tot stand is gekomen. Oorspronkelijk werd een onderzoek bij WiSEO handmatig uitgevoerd door middel van Spreadsheets (of Excel) waarbij ze alle zoekwoorden en hun zoekvolume manueel overtypten (of kopiëren en plakken) vanuit een zoekwoord tool. Om een makkelijk overzicht te krijgen van alle zoekwoorden werd dit onderverdeeld in verschillende groepen of clusters. 

Vanuit WiSEO kwam de vraag om een bestaande zoekwoord tool aan te passen omdat dit handmatige proces teveel tijd in beslag nam. 

 Als resultaat werd er een tool geprogrammeerd om voor zoekwoorden gerelateerde zoekwoorden (of zoekwoordsuggesties) te krijgen en deze makkelijk (binnen het zelfde scherm) in een lijst te stoppen (Deze lijst kan je onderverdelen in meerdere groepen of clusters). De lijst kan dan worden geëxporteerd naar een Excel bestand om het zoekwoord onderzoek af te leveren aan de klant. De reden dat er gewerkt wordt met een Excel bestand is omdat WiSEO altijd op deze manier een zoekwoord onderzoek aflevert aan de klant. 

Als conclusie kunnen we stellen dat er nog uitbreiding mogelijk is voor sommige tools zoals deze zoekwoord tool.

Door dit onderzoek zal er meer duidelijkheid zijn welke SEO-tools best in de toekomst kunnen worden gebruikt door WiSEO en eventueel andere digitale marketing bedrijven. De verbeterde zoekwoord tool zal ook gebruikt worden door WiSEO om een zoekwoordonderzoek vlotter te laten verlopen. 
