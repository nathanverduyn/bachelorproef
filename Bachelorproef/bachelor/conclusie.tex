\chapter{Conclusie}
\label{ch:conclusie}

Voor de vergelijkende studie tussen SEO-tools kan er besloten worden dat er per categorie (zoekwoord tools, ranking tools, linkbuilding tools en technische tools) een beste gratis tool (op vlak van prestatie) en betalende tool (op vlak van prestatie en prijs) is.

Het kiezen van de juiste (en beste) SEO-tool kan de resultaten bevorderen voor de klant, en eventueel tegen een goedkopere prijs (van de tool). 

De vergelijkende studie heeft WiSEO (en eventuele andere digitale marketingbedrijven) meer inzicht gegeven in welke tools nu het best gebruikt kunnen worden per categorie. 

In het tweede deel van het onderzoek werd besproken waarom het nodig is om een keywordtool te programmeren. Er werd er een tool geprogrammeerd om voor bepaalde zoekworden andere gerelateerde zoekwoorden (of zoekwoordsuggesties) te verkrijgen en deze makkelijk (binnen het zelfde scherm) in een lijst te samen te voegen (deze lijst kan je onderverdelen in meerdere groepen of clusters). Die
lijst kan vervolgens geëxporteerd worden naar een Excel-bestand om het zoekwoordonderzoek af te leveren aan de klant.

De verbeterde zoekwoordtool zal ook gebruikt worden door WiSEO om een zoekwoordonderzoek vlotter te laten verlopen. Als conclusie kunnen we stellen dat er nog veel uitbreiding mogelijk is voor sommige tools zoals deze zoekwoordtool. 